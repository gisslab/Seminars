\documentclass[notes,11pt, aspectratio=169]{beamer}

\usepackage{pgfpages}
% These slides also contain speaker notes. You can print just the slides,
% just the notes, or both, depending on the setting below. Comment out the want
% you want.
\setbeameroption{hide notes} % Only slide
%\setbeameroption{show only notes} % Only notes
% \setbeameroption{show notes on second screen=right} % Both

\usepackage{import}
\usepackage{helvet}
% \usepackage[default]{lato}
\usepackage{array}
% \usepackage{natbib}
% \bibliographystyle{plain}
\usepackage{apalike}
\bibliographystyle{apalike}
% \usepackage[natbib, maxcitenames=3, mincitenames=11, style=apa]{biblatex}

\usepackage{tikz}
\newcommand*\circled[4]{\tikz[baseline=(char.base)]{
    \node[shape=circle, fill=#2, draw=#3, text=#4, inner sep=2pt] (char) {#1};}}
\usepackage{verbatim}
\setbeamertemplate{note page}{\pagecolor{yellow!5}\insertnote}
\usetikzlibrary{positioning}
\usetikzlibrary{snakes}
\usetikzlibrary{calc}
\usetikzlibrary{arrows}
\usetikzlibrary{decorations.markings}
\usetikzlibrary{shapes.misc}
\usetikzlibrary{matrix,shapes,arrows,fit,tikzmark}
\usepackage{amsmath}
\usepackage{mathpazo}
\usepackage{hyperref}
\usepackage{lipsum}
\usepackage{multimedia}
\usepackage{graphicx}
\usepackage{multirow}
\usepackage{graphicx}
\usepackage{dcolumn}
\usepackage{bbm}
\usepackage{cancel}
\newcolumntype{d}[0]{D{.}{.}{5}}
\usepackage{subcaption}
\usepackage{changepage}
\usepackage{appendixnumberbeamer}
\newcommand{\beginbackup}{
   \newcounter{framenumbervorappendix}
   \setcounter{framenumbervorappendix}{\value{framenumber}}
   \setbeamertemplate{footline}
   {
     \leavevmode%
     \hline
     box{%
       \begin{beamercolorbox}[wd=\paperwidth,ht=2.25ex,dp=1ex,right]{footlinecolor}%
%         \insertframenumber  \hspace*{2ex} 
       \end{beamercolorbox}}%
     \vskip0pt%
   }
 }
\newcommand{\backupend}{
   \addtocounter{framenumbervorappendix}{-\value{framenumber}}
   \addtocounter{framenumber}{\value{framenumbervorappendix}} 
}


\usepackage{graphicx}
\usepackage[space]{grffile}
\usepackage{booktabs}

% These are my colors -- there are many like them, but these ones are mine.
\definecolor{blue}{RGB}{20,160,210}
\definecolor{red}{RGB}{213,94,0}
\definecolor{yellow}{RGB}{240,228,66}
\definecolor{green}{RGB}{0,158,115}

% % Enviroments
% \newtheorem{defin}{Definition.}
% \newtheorem{teo}{Theorem. }
% \newtheorem{lema}{Lemma. }
% \newtheorem{coro}{C
% \begin{frame}{Modeling Choice}orolary. }
% \newtheorem{prop}{Proposition. }
% \theoremstyle{definition}
% \newtheorem{examp}{Example. }
% % \numberwithin{problem}{subsection} 

\hypersetup{
  colorlinks=false,
  linkbordercolor = {white},
  linkcolor = {blue}
}


%% I use a beige off white for my background
\definecolor{MyBackground}{RGB}{255,253,218}

%% Uncomment this if you want to change the background color to something else
%\setbeamercolor{background canvas}{bg=MyBackground}

%% Change the bg color to adjust your transition slide background color!
\newenvironment{transitionframe}{
  \setbeamercolor{background canvas}{bg=white}
  \begin{frame}}{
    \end{frame}
}

\setbeamercolor{frametitle}{fg=blue}
\setbeamercolor{title}{fg=black}
\setbeamertemplate{footline}[frame number]
\setbeamertemplate{navigation symbols}{} 
%\setbeamertemplate{itemize items}{-}
\setbeamercolor{itemize item}{fg=blue}
\setbeamercolor{itemize subitem}{fg=blue}
\setbeamercolor{enumerate item}{fg=blue}
\setbeamercolor{enumerate subitem}{fg=blue}
\setbeamercolor{button}{bg=MyBackground,fg=blue,}
\setbeamercolor{theotem}{fg=blue} 

% If you like road maps, rather than having clutter at the top, have a roadmap show up at the end of each section 
% (and after your introduction)
% Uncomment this is if you want the roadmap!
%\AtBeginSection[]
%{
%   \begin{frame}
%       \frametitle{Roadmap of Talk}
%       \tableofcontents[currentsection]
%   \end{frame}
%}


\setbeamercolor{section in toc}{fg=blue}
\setbeamercolor{subsection in toc}{fg=red}
\setbeamersize{text margin left=1em,text margin right=1em} 

\newenvironment{wideitemize}{\itemize\addtolength{\itemsep}{10pt}}{\enditemize}

\usepackage{environ}
\NewEnviron{videoframe}[1]{
  \begin{frame}
    \vspace{-8pt}
    \begin{columns}[onlytextwidth, T] % align columns
      \begin{column}{.58\textwidth}
        \begin{minipage}[t][\textheight][t]
          {\dimexpr\textwidth}
          \vspace{8pt}
          \hspace{4pt} {\Large \sc \textcolor{blue}{#1}}
          \vspace{8pt}
          
          \BODY
        \end{minipage}
      \end{column}%
      \hfill%
      \begin{column}{.42\textwidth}
        \colorbox{green!20}{\begin{minipage}[t][1.2\textheight][t]
            {\dimexpr\textwidth}
            Face goes here
          \end{minipage}}
      \end{column}%
    \end{columns}
  \end{frame}
}

\title[]{\textcolor{blue}{The Reserve Supply Channel of Unvonventional Monetary Policy \\ Diamond, Jiang, Ma (2022)}}

\author{ Giselle Labrador Badia}
%\centering
\date{\today}


\begin{document}
%%% TIKZ STUFF
\tikzset{   
        every picture/.style={remember picture,baseline},
        every node/.style={anchor=base,align=center,outer sep=1.5pt},
        every path/.style={thick},
        }
\newcommand\marktopleft[1]{%
    \tikz[overlay,remember picture] 
        \node (marker-#1-a) at (-.3em,.3em) {};%
}
\newcommand\markbottomright[2]{%
    \tikz[overlay,remember picture] 
        \node (marker-#1-b) at (0em,0em) {};%
}
\tikzstyle{every picture}+=[remember picture] 
\tikzstyle{mybox} =[draw=black, very thick, rectangle, inner sep=10pt, inner ysep=20pt]
\tikzstyle{fancytitle} =[draw=black,fill=red, text=white]
%%%% END TIKZ STUFF

% % Title Slide
\begin{frame}
  \maketitle
\end{frame}
% % % Outline Slide
% \begin{frame}
%   \frametitle{Roadmap of Talk}
%   \tableofcontents
% \end{frame}

% INTRO
% \section{Motivation}

% \begin{frame}{Motivation}

% %\textbf{Preventive care}
% \begin{itemize}

%     \item 
%         \vspace{0.1cm}
%     \item 
    
% \end{itemize}

% \end{frame}


% \begin{frame}{Motivation}

%   \begin{figure}[t*]
%   %   \begin{subfigure}[t]{0.45\textwidth}
%   %     \label{fig:fig1}
%   %   \includegraphics[width=1\textwidth]{/Users/gisellelab/Work/deposit_pricing/document/imgs/boa_2010.pdf}
%   %   \caption{Bank of America}
%   %   \end{subfigure}\hfill
  
%   %   \begin{subfigure}[t]{0.45\textwidth}
%   %     \label{fig:fig2}
%     \includegraphics[width=.4\textwidth]{/Users/gisellelab/Work/deposit_pricing/document/imgs/wf_2010.pdf}
%     \caption{Wells Fargo}
%   %   \end{subfigure}
%   \end{figure}
  
%   \end{frame}
  
   
  \begin{frame}{Motivation}
    \begin{itemize}
\item Banks set different deposit rates by geographic area. 

\item When markets have different degrees of competition, the effects on banks profits and welfare are ambiguous. 


    
    \end{itemize}

    
    \begin{figure}[t*]
      \begin{subfigure}[t]{0.45\textwidth}
        \label{fig:fig1}
      \includegraphics[width=1\textwidth]{/Users/gisellelab/Work/deposit_pricing/document/imgs/boa_2010.pdf}
      \caption{Bank of America}
      \end{subfigure}\hfill
      \begin{subfigure}[t]{0.45\textwidth}
        \label{fig:fig2}
      \includegraphics[width=1\textwidth]{/Users/gisellelab/Work/deposit_pricing/document/imgs/wf_2010.pdf}
      \caption{Wells Fargo}
      \end{subfigure}
    \end{figure}
    % higher rate is green light
    \end{frame}
    
  
\begin{frame}{Research Question}

  %{\textcolor{blue}{\large{Research question}}}
  

\begin{itemize}
    \item  What is the impact of deposit rate zoning on rate dispersion and profits?
\end{itemize}
\vspace{1cm}
Objectives:
\begin{itemize}
  \item Document asymmetries in deposit pricing strategies in the US retail banking. 
  \item Examine implications of deposit price zoning on rate dispersion and profits.
\end{itemize}

\end{frame}

\begin{frame}{Motivation}

  \textbf{What should we care?}
  \vspace{0.3cm}
  \begin{itemize}
  
     \item US deposits in banks were over \$18 trillion in 2021.
      \item Deposits are 75 \% of the funding of US commercial banks. 
          \vspace{0.1cm}
      \item 82 \% of total deposits come from personal and non-financial business.
          \vspace{0.1cm}
    
      \item An estimated 94.6 \% of U.S. households have a checking or saving account (2019)
      \vspace{0.1cm}
      \item Depositors care about competitive rates and physical presence.
  \end{itemize}


\end{frame}


\section{Related Literature}
\begin{frame}[label = lit]{Related Literature}



  \begin{wideitemize}
      \item Zone pricing
      
      \begin{itemize}
        \item Supermarket chains: Chintagunta, Dubé, and Singh (2003)
        \item Retail home-improvement: Adams and Williams (2019)
      \end{itemize}


          \item Structural models of deposit competition

          \begin{itemize}
            \item Dick(2008)
            \item Ho and Ishii (2011)
            \item Aguirregabiria, Clark, and Wang (2016, 2017)
          \end{itemize}
      
      \item Uniform pricing 
      
      \begin{itemize}
          \item Chains: DellaVigna and Gentzkow (2019)
          \item Bank deposits: Granja and Paixao (2019), Begenau and Stafford (2022)
      \end{itemize}

  \end{wideitemize}
  
\end{frame}

% \section{Background and Data}


% \begin{frame}[label = backg]{Background}



% \begin{itemize}
%     \item 
    
%     \item  \hyperlink{backg_col_diagram_appendix}{\beamergotobutton{C}}
%     \begin{itemize}
%         \item 
%     \end{itemize}
    
%     \item 
%     \end{itemize}

% \end{frame}


\begin{frame}{Data}

  Main data sources:
  \vspace{0.3cm}
\begin{wideitemize}
    \item  Summary of Deposits by FDIC (SOD)
    
    \begin{itemize}
      \item yearly as of June 30th. 
      \item commercial banks and thrifts FDIC insured
        \item quantity of deposits by branch 
        \item location by branch
    \end{itemize}
    \vspace{0.3cm}
    \item RateWatch (RW)
    
    \begin{itemize}
        \item Weakly survey over 100000 branches
        \item commercial banks, credit unions, thrifts
        \item Rates (APR and APY), FDIC identifier, geographic information
        \item products (e.g. savings, CD, MM, etc.) by size and maturity (12-month CD min 10K)
        \item If the branch is rate setter
    \end{itemize}
    \item 

    \end{wideitemize}

\end{frame}

    \begin{frame}{Data}

    Additional data sources:
    \vspace{0.3cm}
    \begin{wideitemize}
      \item Call Reports and Thrift Financial Reports (Call Reports)
        
      \begin{itemize}
          \item income statement, balance sheet, loans, deposits, investments,  bank's capital, asset sale information
          \item bank characteristics: number of employees
        
          
      \end{itemize}

      \item The Financial Performance Reports and National Credit Union Association
        \begin{itemize}
          \item deposits 
      \end{itemize}
    \end{wideitemize}

  \end{frame}
% !!! HERE
  \begin{frame}{Data}\label{data}
  
    Limitations of the data:

    \begin{wideitemize}
      \item RW: does not include the universe of commercial banks and thrifts.
      \item SOD: Yearly data, aggregates secure and unsecured deposits, personal and business deposits, and products.
    \end{wideitemize}

    \vspace{0.3cm}

    This paper will focus:
    \begin{itemize}
        \item Aggregate by years(2001-2020), MSA (non-rural markets)
        \item Commercial banks
        \item APY as deposit rate
        \item Product: 
        \begin{itemize}
          \item one product, e.g. MM 25K, 12-month CD 10K
          \item family of products, e.g. CD, saving
          \item More details in the \hyperlink{backg_data_appendix}{\beamergotobutton{appendix}}.
        \end{itemize}
    \end{wideitemize}

    \end{frame}

\begin{frame}{Descriptive Statistics}

  After RW and SOD match:
  \begin{table}[h]
  \import{tables/}{general.tex}
  \end{table}

  \begin{itemize}
    \item Including local banks
  \end{itemize}
\end{frame}


\begin{frame}{Descriptive Statistics: Market Level}

  %\textbf{Market level:}

  \begin{table}[h]
  \import{tables/}{mkt_tab.tex}
  \end{table}

  %\vspace(0.5cm)
  \begin{itemize}
    \item Only local banks in outside option
    \item Credit unions are missing from outside option (around 8\% share in 2020)
  \end{itemize}
\end{frame}



\begin{frame}{Data}

  \begin{itemize}
    \item For years above 2004, more than 93\% of the markets have more than 85\% deposit coverage. 
    \item HHI index of matched data set. 
  \end{itemize}
  
    \begin{figure}[t*]
      \centering
    %   \begin{subfigure}[t]{0.45\textwidth}
    %     \label{fig:fig1}
    %   \includegraphics[width=1\textwidth]{/Users/gisellelab/Work/deposit_pricing/document/imgs/boa_2010.pdf}
    %   \caption{Bank of America}
    %   \end{subfigure}\hfill
    
    %   \begin{subfigure}[t]{0.45\textwidth}
    %     \label{fig:fig2}
      \includegraphics[width=.5\textwidth]{/Users/gisellelab/Work/deposit_pricing/document/imgs/hhi.pdf}
      % \caption{Wells Fargo}
    %   \end{subfigure}
    \end{figure}
    
  \end{frame}






\begin{frame}{Model: Supply}
  
Banks:

\begin{itemize}
  \item Banks minimize cost of deposit funding:

\end{itemize}

\begin{equation}
  \label{eq:empirical-prev3}
  \pi_{j} =  l_j D_j s_j(\cdot) - \sum_{t=1}^M_j (r_{jt} + cm_{jt})s_{jt}(r_{jt}, r_{-j,t}D_{jt}\cdot) 
\end{equation}


\begin{itemize}
  \item $r_{jt}$ is deposit rate of bank $j$ in market $t$
  \item $D_{jt}$ is total deposits of bank $j$ in market $t$
  \item $l_j$ is return on investment for bank $j$
  \item $s_{jt}$ is market share of bank $j$
  \item $cm_{jt}$ is marginal cost

\end{itemize}

\end{frame}


\section{Future Work}

\begin{frame}{Future Work}
        \begin{wideitemize}
        \item Clean the remaining data sources (Call Reports and Credit Union)

        \item Document price dispersion.
        
        \item Estimate model.
        
        \item Find instruments.
        
        \item Counterfactual analysis
         \begin{itemize}
          \item Price discrimination by markets
          \item Single price zone 
          \item Need return on assets by banks
        \end{itemize}

        \item Improve demand for deposits, where consumers are endowed with a deposit ($income$):
        $$\alpha dep_{it} r_{jt}$$

        \end{wideitemize}
    
\end{frame}

\begin{frame}
\textcolor{blue}{\huge{\centerline{Thank you!}}}
\end{frame}

%\section*{References}
%\begin{frame}{References}
%    \bibliographystyle{plainnat}
%    \bibliography{references.bib}
%\end{frame}


\appendix


 \begin{frame}[label=backg_data_appendix]{Appendix: Data Description \hyperlink{data}{\beamergotobutton{Back}}}
\begin{itemize}
  \item Product: SAV 2.5K (second most popular product)
  \item Impute prices:
  \item \begin{enumerate}
    \item RateWatch data is missing some weeks (around 75\% of observations have missing weeks)
    $$r_{jwt} = \text{ avg non-missing }_{jt} + \text{ fed rate$_w$ } + \text{ national avg$_{jw}$,} $$
    \item From network of banks (rate setter) using RW.
    \item Closest branch to the market centroid. 
  \end{enumerate}
  \item Drop bank-markets observations with less than 1\% rates.
  \item Drop bank-markets observations with a single branch.
  \item Local banks are defined as banks that get more than 90\% deposits from single market. 
\end{itemize}
 \end{frame}


% \begin{frame}[label=backg_col_brack_appendix]{Appendix\hyperlink{backg}{\beamergotobutton{Back}}}

% \end{frame}

\end{document}