\documentclass[notes,11pt, aspectratio=169]{beamer}

\usepackage{pgfpages}
% These slides also contain speaker notes. You can print just the slides,
% just the notes, or both, depending on the setting below. Comment out the want
% you want.
\setbeameroption{hide notes} % Only slide
%\setbeameroption{show only notes} % Only notes
% \setbeameroption{show notes on second screen=right} % Both

\usepackage{import}
\usepackage{helvet}
% \usepackage[default]{lato}
\usepackage{array}
% \usepackage{natbib}
% \bibliographystyle{plain}
\usepackage{apalike}
\bibliographystyle{apalike}
% \usepackage[natbib, maxcitenames=3, mincitenames=11, style=apa]{biblatex}

\usepackage{tikz}
\newcommand*\circled[4]{\tikz[baseline=(char.base)]{
    \node[shape=circle, fill=#2, draw=#3, text=#4, inner sep=2pt] (char) {#1};}}
\usepackage{verbatim}
\setbeamertemplate{note page}{\pagecolor{yellow!5}\insertnote}
\usetikzlibrary{positioning}
\usetikzlibrary{snakes}
\usetikzlibrary{calc}
\usetikzlibrary{arrows}
\usetikzlibrary{decorations.markings}
\usetikzlibrary{shapes.misc}
\usetikzlibrary{matrix,shapes,arrows,fit,tikzmark}
\usepackage{amsmath}
\usepackage{mathpazo}
\usepackage{hyperref}
\usepackage{lipsum}
\usepackage{multimedia}
\usepackage{graphicx}
\usepackage{multirow}
\usepackage{graphicx}
\usepackage{dcolumn}
\usepackage{bbm}
\usepackage{cancel}
\newcolumntype{d}[0]{D{.}{.}{5}}
\usepackage{subcaption}
\usepackage{changepage}
\usepackage{appendixnumberbeamer}
\newcommand{\beginbackup}{
   \newcounter{framenumbervorappendix}
   \setcounter{framenumbervorappendix}{\value{framenumber}}
   \setbeamertemplate{footline}
   {
     \leavevmode%
     \hline
     box{%
       \begin{beamercolorbox}[wd=\paperwidth,ht=2.25ex,dp=1ex,right]{footlinecolor}%
%         \insertframenumber  \hspace*{2ex} 
       \end{beamercolorbox}}%
     \vskip0pt%
   }
 }
\newcommand{\backupend}{
   \addtocounter{framenumbervorappendix}{-\value{framenumber}}
   \addtocounter{framenumber}{\value{framenumbervorappendix}} 
}


\usepackage{graphicx}
\usepackage[space]{grffile}
\usepackage{booktabs}

% These are my colors -- there are many like them, but these ones are mine.
% \definecolor{blue}{RGB}{20,160,210}
\definecolor{blue}{RGB}{80,150,170}
\definecolor{red}{RGB}{213,94,0}
\definecolor{yellow}{RGB}{240,228,66}
\definecolor{green}{RGB}{0,158,115}

% % Enviroments
% \newtheorem{defin}{Definition.}
% \newtheorem{teo}{Theorem. }
% \newtheorem{lema}{Lemma. }
% \newtheorem{coro}{C
% \begin{frame}{Modeling Choice}orolary. }
% \newtheorem{prop}{Proposition. }
% \theoremstyle{definition}
% \newtheorem{examp}{Example. }
% % \numberwithin{problem}{subsection} 

\hypersetup{
  colorlinks=false,
  linkbordercolor = {white},
  linkcolor = {blue}
}


%% I use a beige off white for my background
\definecolor{MyBackground}{RGB}{255,253,218}

%% Uncomment this if you want to change the background color to something else
%\setbeamercolor{background canvas}{bg=MyBackground}

%% Change the bg color to adjust your transition slide background color!
\newenvironment{transitionframe}{
  \setbeamercolor{background canvas}{bg=white}
  \begin{frame}}{
    \end{frame}
}

\setbeamercolor{frametitle}{fg=blue}
\setbeamercolor{title}{fg=black}
\setbeamertemplate{footline}[frame number]
\setbeamertemplate{navigation symbols}{} 
%\setbeamertemplate{itemize items}{-}
\setbeamercolor{itemize item}{fg=blue}
\setbeamercolor{itemize subitem}{fg=blue}
\setbeamercolor{enumerate item}{fg=blue}
\setbeamercolor{enumerate subitem}{fg=blue}
\setbeamercolor{button}{bg=MyBackground,fg=blue,}
\setbeamercolor{theotem}{fg=blue} 

% If you like road maps, rather than having clutter at the top, have a roadmap show up at the end of each section 
% (and after your introduction)
% Uncomment this is if you want the roadmap!
%\AtBeginSection[]
%{
%   \begin{frame}
%       \frametitle{Roadmap of Talk}
%       \tableofcontents[currentsection]
%   \end{frame}
%}


\setbeamercolor{section in toc}{fg=blue}
\setbeamercolor{subsection in toc}{fg=red}
\setbeamersize{text margin left=1em,text margin right=1em} 

\newenvironment{wideitemize}{\itemize\addtolength{\itemsep}{10pt}}{\enditemize}

\usepackage{environ}
\NewEnviron{videoframe}[1]{
  \begin{frame}
    \vspace{-8pt}
    \begin{columns}[onlytextwidth, T] % align columns
      \begin{column}{.58\textwidth}
        \begin{minipage}[t][\textheight][t]
          {\dimexpr\textwidth}
          \vspace{8pt}
          \hspace{4pt} {\Large \sc \textcolor{blue}{#1}}
          \vspace{8pt}
          
          \BODY
        \end{minipage}
      \end{column}%
      \hfill%
      \begin{column}{.42\textwidth}
        \colorbox{green!20}{\begin{minipage}[t][1.2\textheight][t]
            {\dimexpr\textwidth}
            Face goes here
          \end{minipage}}
      \end{column}%
    \end{columns}
  \end{frame}
}

\title[]{\textcolor{blue}{The Reserve Supply Channel of Unconventional Monetary Policy \\ Diamond, Jiang, Ma (2022)}}

\author{ Presenter: Giselle Labrador Badia}
%\centering
\date{\today}


\begin{document}
%%% TIKZ STUFF
\tikzset{   
        every picture/.style={remember picture,baseline},
        every node/.style={anchor=base,align=center,outer sep=1.5pt},
        every path/.style={thick},
        }
\newcommand\marktopleft[1]{%
    \tikz[overlay,remember picture] 
        \node (marker-#1-a) at (-.3em,.3em) {};%
}
\newcommand\markbottomright[2]{%
    \tikz[overlay,remember picture] 
        \node (marker-#1-b) at (0em,0em) {};%
}
\tikzstyle{every picture}+=[remember picture] 
\tikzstyle{mybox} =[draw=black, very thick, rectangle, inner sep=10pt, inner ysep=20pt]
\tikzstyle{fancytitle} =[draw=black,fill=red, text=white]
%%%% END TIKZ STUFF

% % Title Slide
\begin{frame}
  \maketitle
\end{frame}
% % % Outline Slide
% \begin{frame}
%   \frametitle{Roadmap of Talk}
%   \tableofcontents
% \end{frame}

% INTRO
% \section{Motivation}


% \begin{frame}{Motivation}

%   \begin{figure}[t*]
%   %   \begin{subfigure}[t]{0.45\textwidth}
%   %     \label{fig:fig1}
%   %   \includegraphics[width=1\textwidth]{/Users/gisellelab/Work/deposit_pricing/document/imgs/boa_2010.pdf}
%   %   \caption{Bank of America}
%   %   \end{subfigure}\hfill
  
%   %   \begin{subfigure}[t]{0.45\textwidth}
%   %     \label{fig:fig2}
%     \includegraphics[width=.4\textwidth]{/Users/gisellelab/Work/deposit_pricing/document/imgs/wf_2010.pdf}
%     \caption{Wells Fargo}
%   %   \end{subfigure}
%   \end{figure}
  
%   \end{frame}
  
   


    
\begin{frame}{Motivation}
\begin{wideitemize}
    \item Expansion of central bank reserves issued by the Federal Reserve in last years. 
    \item Federal Reserve purchased trillions of dollars in assets in its Quantitative Easing (QE) program.
    \begin{wideitemize}
        \item Buys securities from banks and pays with reserves that can only be held in the banking system
    \end{wideitemize}
    \item Net injection of trillions of dollars to bank balance sheets
    
    \item \textbf{This paper studies the impact of this large reserve supply on bank borrowing and lending.}

\end{wideitemize}
    
\end{frame}

\begin{frame}{Motivation}

    \begin{wideitemize}
    \item The impact of increasing the reserve supply on bank lending is ambiguous: 
    \vspace{0.15cm}
    \begin{wideitemize}
        \item[$\uparrow$] \textit{lending:} \\Reduce costs of selling illiquid assets in a bank run (Diamond and Dybvig, 1983), \\
         help comply with liquidity regulations.  
        \item[$\downarrow$] \textit{lending:} \\ cost of meeting capital requirements when equity is scarce (Kashyap and Stein 1993),  \\
        amplify liquidity strains during stress episodes (Acharya and Rajan, 2021),\\
        bank leverage regulation can make it costly to expand asset holding (Du, Tepper, and Verdelhan, 2018)    
    \end{wideitemize}
    
    \end{wideitemize}
    
    \end{frame}
    

\begin{frame}{Motivation}
    \vspace{0.5cm}
      \begin{itemize}
        \item  Reserves in balance sheets increased from 50 billion (2008) to 2.8 trillion (2015).
        \item The proportion of illiquid assets on bank balance sheets declined from 83\% to 63\%.
      \end{itemize}
      
        \begin{figure}[t*]
          \centering
    
          \includegraphics[width=.5\textwidth]{./imgs/motiv_supply_reserves_assets.png}
        \caption{Supply of Central Banks Reserves and Bank Asset Illiquidity}
        \end{figure}
        
      \end{frame}
    


\begin{frame}{Goal and Results}

    \vspace{0.5cm}
      \begin{wideitemize}

      \item Estimate structural model of the market for bank deposits and loans:
      \begin{itemize}
        \item Elasticity of deposit and loan demands.
        \item How does holding reserves change the cost of deposits and loans
      \end{itemize}
      \item \textbf{Counterfactual Analysis:} Increase supply of bank reserves and compute new interest rates and quantities. 

\pause

      \item \textbf{Main Findings:} \textbf{The Reserve Supply Channel of QE}
      \begin{itemize}
        \item Demand for bank loans is more interest-rate sensitive than demand for deposits and mortgages. 
        \item Each dollar of reserves injected from 2008 to 2017 crows out 19 cents of bank lending ($\downarrow$ lending)
        \item Deposit and mortgages are less affected 
        \item \textbf{The Reserve Supply Channel of QE}
      \end{itemize}

    \end{wideitemize}
      \end{frame}
    



\section{Related Literature}
\begin{frame}[label = lit]{Related Literature}

  \begin{wideitemize}
      \item Estimate new channel of QE transmission through bank balance sheets. 
      
      \begin{itemize}
        \item Asset pricing (Krishnamurthy and Vissing-Jorgensen, 2011), Bank balance sheets (Rodnyasky and Darmouni, 2017)  
        
      \end{itemize}


          \item Structural models of deposit competition

          \begin{itemize}
            \item Egan, Hortaçsu, and Matvos (2017),  Wang, Whited, Wu and Xiao (2020).
          \end{itemize}
      
      \item Role of imperfect competition in the transmission of conventional monetary policy 
      \begin{itemize}
        \item  Deposits (Drechsler, Savov, and Schnabl, 2017), mortgages (Scharhstein and Sunderam, 2016)
      \end{itemize}
      

      \item Quantify synergies between illiquid loans, liquid securities, and deposit liabilities on bank balance sheets. 
      
      \begin{itemize}
        \item Kashyap and Stein (1993), Du, Tepper, and Verdelhan (2018), Diamond and Rajan (2000)
    \end{itemize}
      

  \end{wideitemize}
  
\end{frame}

% \section{Background and Data}


% \begin{frame}[label = backg]{Background}



% \begin{itemize}
%     \item 
    
%     \item  \hyperlink{backg_col_diagram_appendix}{\beamergotobutton{C}}
%     \begin{itemize}
%         \item 
%     \end{itemize}
    
%     \item 
%     \end{itemize}

% \end{frame}






%     \begin{frame}{Data}

%     Additional data sources:
%     \vspace{0.3cm}
%     \begin{wideitemize}
%       \item Call Reports and Thrift Financial Reports (Call Reports)
        
%       \begin{itemize}
%           \item income statement, balance sheet, loans, deposits, investments,  bank's capital, asset sale information
%           \item bank characteristics: number of employees
        
          
%       \end{itemize}

%       \item The Financial Performance Reports and National Credit Union Association
%         \begin{itemize}
%           \item deposits 
%       \end{itemize}
%     \end{wideitemize}

%   \end{frame}


% \begin{frame}{Descriptive Statistics}

%   After RW and SOD match:
%   \begin{table}[h]
%   \import{tables/}{general.tex}
%   \end{table}

%   \begin{itemize}
%     \item Including local banks
%   \end{itemize}
% \end{frame}


% \begin{frame}{Descriptive Statistics: Market Level}

%   %\textbf{Market level:}

%   \begin{table}[h]
%   \import{tables/}{mkt_tab.tex}
%   \end{table}

%   %\vspace(0.5cm)
%   \begin{itemize}
%     \item Only local banks in outside option
%     \item Credit unions are missing from outside option (around 8\% share in 2020)
%   \end{itemize}

% \end{frame}




\begin{frame}
    \textcolor{blue}{\huge{\centerline{Model of Bank Balance Sheets}}}

\end{frame}

\begin{frame}{Model: Graphical Illustration}
    \vspace{0.5cm}
      \begin{itemize}
        \item Banks' holding of liquid reserves may impact their marginal cost of lending. 
    
      \end{itemize}
      
        \begin{figure}[t*]
          \centering
    
          \includegraphics[width=.5\textwidth]{./imgs/illustrate_effect.png}
        \caption{Effect of increase in reserves in the loan market. }
        \end{figure}
        
      \end{frame}
    


% \begin{frame}
%     \textcolor{blue}{\huge{\centerline{Model of Bank Balance Sheets}}}

% \end{frame}


\begin{frame}{Model}
    \vspace{0.3cm}
    Bank $m$ choose rates and security quantities at time $t$ to maximize the expected present value of its profits at time $t+1$ in all markets $n$:
    \begin{equation}
        \begin{gathered}
        \max _{\left(R_{D, n m t}, R_{M, n m t}, R_{L, n m t}, Q_{S, m t}\right)} \sum_n Q_{L, n m t}\left(R_{L, n m t}-R_t^{L, m}\right)+\sum_n Q_{M, n m t}\left(R_{M, n m t}-R_t^{L, m}\right) \\
        +Q_{S, m t}\left(R_{S, t}-R_t^{S, m}\right)-\sum_n Q_{D, n m t}\left(R_{D, n m t}-R_t^{D, m}\right)-C\left(\Theta_{m t}\right)
        \end{gathered}
        \end{equation}
        \vspace{0.1cm}
        \begin{itemize}
            \item $R_{D,nmt}$, $R_{M,nmt}$, $R_{L,nmt}$ are bank deposit, mortgages, and loan rates.
            \item $R_{S,t}$ is the security rate in the competitive market. 
            \item $R^{D,m}_t$, $R^{M,m}_t$, $R^{L,m}_t$, $R^{S,m}_t$ are the cash flows discount rates of the deposits, mortgages, loans, and securities.
            \item $Q_{D, n m t}$,  $Q_{M, n m t}$,$Q_{L, n m t}$, $Q_{S, m t}$ are the quantities of deposits, mortgages, loans, and securities.
            \item $C(\Theta_{mt})$ is the balance sheet costs ($\Theta_{mt}$ is the vector ($Q_{L, n m t}$, $Q_{M, n m t}$, $Q_{S, m t}$, $Q_{D, n m t}$))
    
            
        \end{itemize}
    \end{frame}
    
\begin{frame}{Model}

    The first order conditions of bank profits with respect to the choice variables, $R_{D, n m t}, R_{M, n m t}, R_{L, n m t}$, and $Q_{S, m t}$, are
    $$
    \begin{aligned}
        R_t^{D, m}-R_{D, n m t}-\frac{Q_{D, n m t}}{\partial Q_{D, n m t} / \partial R_{D, n m t} }& =  \frac{\partial C\left(\Theta_{m t}\right)}{\partial Q_{D, n m t}} \\
        R_{j, n m t}- R_t^{j, m} +\frac{Q_{j, n m t}}{\partial Q_{j, n m t} / \partial R_{j, n m t}} & =  \frac{\partial C\left(\Theta_{m t}\right)}{\partial Q_{j, n m t}}, \quad j \in \{ L,M,S \}\\
    % \overbrace{\frac{\partial}{\partial R_{D, n m t}}\left[Q_{D, n m t}\left(R_t^{D, m}-R_{D, n m t}\right)\right]}^{\text {Marginal Revenue }} & =\overbrace{\frac{\partial C\left(\Theta_{m t}\right)}{\partial Q_{D, n m t}} \frac{\partial Q_{D, n m t}}{\partial R_{D, n m t}}}^{\text {Marginal Cost }}, \\
    % \frac{\partial}{\partial R_{j, n m t}}\left[Q_{j, n m t}\left(R_{j, n m t}-R_t^{j, m}\right)\right] & =\frac{\partial C\left(\Theta_{m t}\right)}{\partial Q_{j, n m t}} \frac{\partial Q_{j, n m t}}{\partial R_{j, n m t}}, \quad j \in \{ L,M\} \} \\
    % \frac{\partial}{\partial R_{L, n m t}}\left[Q_{L, n m t}\left(R_{L, n m t}-R_t^{L, m}\right)\right] & =\frac{\partial C\left(\Theta_{m t}\right)}{\partial Q_{L, n m t}} \frac{\partial Q_{L, n m t}}{\partial R_{L, n m t}}, \\
    \underbrace{R_{S, t}-R_t^{S, m}}_{\text{Reserve spread}} & =\frac{\partial C\left(\Theta_{m t}\right)}{\partial Q_{S, m t}}.
    \end{aligned}
    $$

\end{frame}

\begin{frame}{Model}


    The comparative statics with respect to a change in bank $m$ 's liquid security holdings $Q_{S, m t}$ are
$$
\begin{aligned}
\frac{\partial\left(R_t^{D, m}-R_{D, n m t}-\frac{Q_{D, n m t}}{\partial Q_{D, n m t} / \partial R_{D, n m t}}\right)}{\partial Q_{D, n m t}} \frac{\partial Q_{D, n m t}}{\partial Q_{S, m t}} & =\frac{\partial^2 C\left(\Theta_{m t}\right)}{\partial Q_{D, n m t} \partial \Theta_{m t}} \cdot \frac{\partial \Theta_{m t}}{\partial Q_{S, m t}} \\
\frac{\partial\left(R_t^{j, m}-R_{j, n m t}-\frac{Q_{j, n m t}}{\partial Q_{j, n m t} / \partial R_{j, n m t}}\right)}{\partial Q_{j, n m t}} \frac{\partial Q_{j, n m t}}{\partial Q_{S, m t}} & =-\frac{\partial^2 C\left(\Theta_{m t}\right)}{\partial Q_{j, n m t} \partial \Theta_{m t}} \cdot \frac{\partial \Theta_{m t}}{\partial Q_{S, m t}}, \quad j \in \{ L,M\} \}  \\
% \frac{\partial\left(R_t^{L, m}-R_{L, n m t}-\frac{Q_{L, n m t}}{\partial Q_{L, n m t} / \partial R_{L, n m t}}\right)}{\partial Q_{L, n m t}} \frac{\partial Q_{L, n m t}}{\partial Q_{S, m t}} & =-\frac{\partial^2 C\left(\Theta_{m t}\right)}{\partial Q_{L, n m t} \partial \Theta_{m t}} \cdot \frac{\partial \Theta_{m t}}{\partial Q_{S, m t}} \\
% \frac{\partial Q_{S, m t}}{\partial Q_{S, m t}} & =1
\end{aligned}
$$
 where $\frac{\partial Q_{j, n m t}}{\partial Q_{S, m t}}$ is the response of each bank branch quantity of $j \in \{D, L, M \}$ % to a change in $Q_{S, m t}$, and $\frac{\partial \Theta_{m t}}{\partial Q_{S, m t}}$ is the response of the balance sheet costs to a change in $Q_{S, m t}$.
\end{frame}


\begin{frame}
    \textcolor{blue}{\huge{\centerline{Demand system}}}
\end{frame}

\begin{frame}{Demand system}
    \vspace{0.2cm}
    \begin{wideitemize}
\item Annual bank-market-level data from 2001 to 2017
\item  Deposits
\begin{itemize}
    \item Deposit quantities: FDIC
    \item Deposit rate: RateWatch (10K Money Market rate)
   \item County-level market
   \end{itemize}
\item  Mortgages
\begin{itemize}
    \item  Mortgage quantities: HMDA
    \item Mortgage rate: RateWatch (15-Year Fixed Rate)
     \item  County-level market
\end{itemize}
\item Loans:
\begin{itemize}
\item Loan quantities and rates: Dealscan
\item State-level market (defined by location of the borrower)
\end{itemize}
\item Bank-level characteristics from Call Reports
    \end{wideitemize}
\end{frame}

\begin{frame}{Demand system: Descriptives}
    \vspace{0.5cm}
    %   \begin{itemize}
    %     \item  bla
    %   \end{itemize}
      
        \begin{figure}[t*]
          \centering
    
          \includegraphics[width=.85\textwidth]{./imgs/summary_statistics.png}
        % \caption{Supply of Central Banks Reserves and Bank Asset Illiquidity}
       
        \end{figure}

        % \hyperlink{firststage}{\beamergotobutton{First Stage Estimates}}
        
      \end{frame}


\begin{frame}{Demand system}
    \vspace{0.2cm}
    \begin{wideitemize}
    \item Depositor $j$ investing in bank $m$ in market $m$ has the fowing utility:
    $$u_{D, j n m t}=\alpha_D R_{D, n m t}+X_{D, n m t} \beta_D+\delta_{D, n m t}+\varepsilon_{D, j n m t}$$.

    where $R_{D, n m t}$ is the deposit rate, $X_{D, n m t}$ is bank characteristics, $\delta_{D, n m t}$ are unobserved characteristics and $\varepsilon_{D, j n m t}$ is the idiosyncratic shock folows T1EV. 



         \item Estimate logit demand system using 2SLS. \\%, where deposit quantities $Q_{D, n m t}$ satisfy the following linear relationship
% $$
% \log Q_{D, n m t}-\log Q_{D, n m^{\prime} t}=\alpha_D\left(R_{D, n m t}-R_{D, n m t}\right)+\beta_D\left(X_{D, n m t}-X_{D, n m^{\prime} t}\right)+\left(\delta_{D, n m t}-\delta_{D, n m^{\prime} t}\right) \text {. }
% $$
% \item Estimate $\alpha_D$ by 2 stage least squares:
% $$
% \begin{aligned}
% R_{D, n m t} & =\gamma_{D, n t}+\gamma_D z_{D, n m t}+X_{D, m t} \gamma_D+e_{D, n m t}, \\
% \log Q_{D, n m t} & =\zeta_{D, n t}+\alpha_D R_{D, n m t}+X_{D, n m t} \beta_D+\delta_{D, n m t} .
% \end{aligned}
% $$
 \item Similar demands for mortgages and loans.

\end{wideitemize}

\end{frame}

\begin{frame}{Demand system}
    \vspace{0.2cm}
    \begin{wideitemize}

\item Market size $\bar{Q}_{D, n t}$:
$$
\bar{Q}_{D, n t}=\bar{F}_{D, n t} \frac{\exp \left(\psi_{D, n t}\right)}{1+\exp \left(\psi_{D, n t}\right)}
$$
 where $\psi_{D, n t}=\log \left(\sum_m \exp \left(\alpha_D R_{D, n m t}+X_{D, n m t} \beta_D+\delta_{D, n m t}\right)\right)$ is the desirability of the composite good.
\item Use a linear approximation $\log \bar{Q}_{D, n t} \approx \log \bar{F}_{D, n t}+\beta_{D, o} \psi_{D, n t}$ to estimate how $Q_{D, n m t}$ responds to changes in $\psi_{D, n t}$.

\end{wideitemize}

\end{frame}




\begin{frame}{Demand system: Natural Disaster Instrument}
    \vspace{0.2cm}
    
    \begin{wideitemize}
% \item - Need: shock to loan/deposit supply to trace out demand curves
\item  Reallocation of bank funding after natural disasters, Cortes and Strahan (2017)
\item Positive shock to local loan demand followed by reallocation of funds away from other branches creates negative loan supply shocks. %at other branches of bank

\item Natural disasters do not directly affect demand for deposits, loans, and mortgages in unaffected counties (in a way that correlated with banks' branch networks).
\pause
\item For bank $m$ in market $n$ in year $t$ :
$$
z_{n m t}=\frac{1}{N_{m t}^u} \log \left(\sum_{n^{\prime}} \text { damage }_{n^{\prime} t} \cdot \frac{Q_{D, n^{\prime} m t}}{\sum_{n_0} Q_{D, n_0 m t}}\right)
$$

\begin{itemize}
\item $N_{m t}^u$ : number of unaffected branches of bank $m$
\item damage $e_{n^{\prime} t}$ : property loss in market $n^{\prime}$
\item $\frac{Q_{D, n^{\prime} m t}}{\sum_{n_0} Q_{D, n_0 m t}}$ : fraction of deposits belonging to branches of bank $m$ in affected markets
\end{itemize}
\end{wideitemize}
\end{frame}


\begin{frame}{Demand system}\label{demand}
    \vspace{0.5cm}
    %   \begin{itemize}
    %     \item  bla
    %   \end{itemize}
      
        \begin{figure}[t*]
          \centering
    
          \includegraphics[width=.85\textwidth]{./imgs/second_stage_demand.png}
        % \caption{Supply of Central Banks Reserves and Bank Asset Illiquidity}
       
        \end{figure}

        \hyperlink{firststage}{\beamergotobutton{First Stage Estimates}}
        
      \end{frame}

      \begin{frame}{Demand system}
        \vspace{0.3cm}
          \begin{itemize}
            \item  Outside option parameters estimates use market-bank-level average instrument. 
            \item Includes county-level control variables like the average age, average income, college education,
            log population, etc. %growth of house prices, log property damage due to natural disasters, and lagged
            quantities.

                       % \hyperlink{firststage}{\beamergotobutton{First Stage Estimates}}
                       \item The increase in deposit quantity when all banks $R_D$ by 10 basis points:
                       $$
                       \frac{\partial \log \bar{Q}_{D, n t}}{\partial R_{D, n t}}=\frac{\partial \log \bar{Q}_{D, n t}}{\partial \psi_{D, n t}^o} \frac{\partial \psi_{D, n t}^o}{\partial R_{D, n t}}=0.28 \times 4.7 \%=1.3 \%
                       $$
          \end{itemize}
          
            \begin{figure}[t*]
              \centering
        
              \includegraphics[width=.41\textwidth]{./imgs/outsideoption.png}
            % \caption{Supply of Central Banks Reserves and Bank Asset Illiquidity}
           
            \end{figure}
    
 
          \end{frame}
    


\begin{frame}
    \textcolor{blue}{\huge{\centerline{Cost function}}}
\end{frame}

\begin{frame}{Cost function}
    \vspace{0.5cm}
For bank $m$ at time $t$ the cost is:

\begin{equation}
    \begin{aligned}
    C\left(\Theta_{m t}\right) & =H\left(Q_{D, m t}, Q_{M, m t}, Q_{L, m t}, Q_{S, m t}\right) \\
    & +\sum_n\left(Q_{M, n m t} \varepsilon_{M, n m t}^Q+Q_{L, n m t} \varepsilon_{L, n m t}^Q+Q_{D, n m t} \varepsilon_{D, n m t}^Q\right)+Q_{S, m t} \varepsilon_{m t}^S
    \end{aligned}
    \end{equation}

    where $H(\cdot)$ is 
    $$
    \begin{aligned}
    & H\left(Q_{D, m t}, Q_{M, m t}, Q_{L, m t}, Q_{S, m t}\right)=\mu_D Q_{D, m t}+\mu_M Q_{M, m t}+\mu_L Q_{L, m t}+\mu_Q Q_{S, m t} \\
    + & \frac{1}{2}\left(K_1 \mathcal{E}_{m t}^2+K_2 \mathcal{I}_{m t}^2+K_3 Q_{D, m t}^2+2 K_4 \mathcal{I}_{m t} Q_{D, m t}+2 K_5 \mathcal{E}_{m t} Q_{D, m t}\right)
    \end{aligned}
    $$
    where $\mathcal{E}_{m t}=Q_{M, m t}+Q_{L, m t}+Q_{S, m t}-Q_{D, m t}$ (bank's equity and non-deposit fund) and $\mathcal{I}_{m t}=Q_{S, m t}+\omega_M Q_{M, m t}+\omega_L Q_{L, m t}$ (liquidity of bank assets).
\end{frame}


    \begin{frame}{Cost function}
        \vspace{0.5cm}

        Diferentiating $C$ with respect to $Q_{D, n m t}$:
    \begin{equation}
        \frac{\partial C}{\partial Q_{D, n m t}}=\mu_D-K_1 \mathcal{E}_{m t}+K_3 Q_{D, m t}+K_4 \mathcal{I}_{m t}+K_5\left(\mathcal{E}_{m t}-Q_{D, m t}\right)+\varepsilon_{n m t}^D
        \end{equation}

        Recall profit maximizer bank's FOC:
        \begin{equation}
            \frac{1}{N_{m t}} \sum_n\left(\frac{\partial C}{\partial Q_{D, n m t}}-R_t\right)=\mu_D^*-K_1 \mathcal{E}_{m t}+K_3 Q_{D, m t}+K_4 \mathcal{I}_{m t}+K_5\left(\mathcal{E}_{m t}-Q_{D, m t}\right)+\varepsilon_{m t}^D
            \end{equation}

        \end{frame}

        \begin{frame}{Cost function}\label{cost}
            \vspace{0.5cm}
    
              \begin{itemize}
    
                \item  $z^1$: Natural disaster shock (bank level)
                \item  $z^2$: Bank's exposure to regional deposit demand shocks (Bartik instrument). Average deposit market growth in counties where the bank has branches.
                \vspace{0.5cm}
                \item  Regress marginal costs of borrowing/lending and all balance sheet quantities on each demand IV: %, e.g., deposits
                $$
                \begin{aligned}
                & C_{D, m t}=\theta_t^D+\kappa^{i, D} z_{m t}^i+u_{D, m t}^Q \\
                & Q_{D, m t}=\alpha_t^D+\gamma^{i, D} z_{m t}^i+\varepsilon_{D, m t}^Q
                \end{aligned}
                $$
                % \item  Regression coefficients jointly determine cost function parameters
        

                \end{itemize}
    \end{frame}

\begin{frame}{Cost function: Estimation}\label{cost}
    \vspace{0.5cm}

      
        \begin{figure}[t*]
          \centering
    
          \includegraphics[width=.8\textwidth]{./imgs/cost_main.png}
        % \caption{Supply of Central Banks Reserves and Bank Asset Illiquidity}
        \end{figure}

        \hyperlink{costiv}{\beamergotobutton{Cost instruments estimates}}

              \begin{itemize}
        \item $\uparrow \$ 100$ million in reserves for each bank branch
        \item $\downarrow 21.9 \mathrm{bps} $ in MC of deposits
      \end{itemize}
        
    \end{frame}

\begin{frame}
    \textcolor{blue}{\huge{\centerline{Counterfactuals}}}
\end{frame}

\begin{frame}{Counterfactuals}
    \vspace{0.5cm}
    \begin{itemize}
        \item  In counterfactual IOER spread increases by an average of 16 bps (11.6 in data).
        \item Average interest rates on deposits, mortgages, and loans increase by
        12.7 bps, 18.8 bps, and 15.6 bps.
        \item  Loans to firms at 19 cents per dollar of reserves
        \item Deposits and mortgages respond less.
      \end{itemize}
      
        \begin{figure}[t*]
          \centering
    
          \includegraphics[width=.9\textwidth]{./imgs/counterf_results.png}
        % \caption{Supply of Central Banks Reserves and Bank Asset Illiquidity}
        \end{figure}
        
      \end{frame}

\begin{frame}{Counterfactuals}
    \vspace{0.5cm}
      \begin{itemize}
        % \item  In counterfactual IOER spread increases by an average of 16 bps (11.6 in data).
        % \item Average interest rates on deposits, mortgages, and loans increase by
        % 12.7 bps, 18.8 bps, and 15.6 bps.
        \item  Estimated impact of reserve supply in bank loan quantities.
      \end{itemize}
      
        \begin{figure}[t*]
          \centering
          \includegraphics[width=.52\textwidth]{./imgs/counterf_figure.png}
        \caption{Reserve Supply and Reduction in Corporate Loan Issuance}
        \end{figure}
        
      \end{frame}


      \begin{frame}
        \textcolor{blue}{\huge{\centerline{Conclusions}}}
    \end{frame}
    

      \begin{frame}{Conclusions}
        \vspace{0.5cm}
        \begin{wideitemize}
        \item Propose reserve supply channel" to quantify the effect of reserve supply on bank balance sheets.
        \item   Estimate structural model:
        \begin{wideitemize}
            \item Demand of deposits, mortgages, and loans
            \item  Supply with cost interactions between bank balance sheet components
            \item  Identification: cross-sectional instruments
        \end{wideitemize}
                        
            \item   $\$ 1$ of reserves crowd out 19 cents of loans from bank balance sheets.
            \item \textbf{Potential solutions:} Relax bank leverage regulation (SLR), allow non-banks to hold reserves.
                    \end{wideitemize}
                    
\end{frame}

\begin{frame}
\textcolor{blue}{\huge{\centerline{Thank you!}}}
\end{frame}

%\section*{References}
%\begin{frame}{References}
%    \bibliographystyle{plainnat}
%    \bibliography{references.bib}
%\end{frame}


\appendix

\begin{frame}[label=firststage]{Appendix: Demand Estimation First Stage \hyperlink{demand}{\beamergotobutton{Back}}}
    \vspace{0.5cm}
    %   \begin{itemize}
    %     \item  bla
    %   \end{itemize}
      
        \begin{figure}[t*]
          \centering
    
          \includegraphics[width=.73\textwidth]{./imgs/first_stage_demand.png}
        % \caption{Supply of Central Banks Reserves and Bank Asset Illiquidity}
        \end{figure}
        
      \end{frame}

      \begin{frame}[label=costiv]{Appendix: Cost Function Instruments Estimates \hyperlink{cost}{\beamergotobutton{Back}}}
        \vspace{0.5cm}
        %   \begin{itemize}
        %     \item  bla
        %   \end{itemize}
          
            \begin{figure}[t*]
              \centering
        
              \includegraphics[width=.85\textwidth]{./imgs/cost_iv_regs.png}
            % \caption{Supply of Central Banks Reserves and Bank Asset Illiquidity}
            \end{figure}
            
          \end{frame}
%  \begin{frame}[label=backg_data_appendix]{Appendix: Data Description \hyperlink{data}{\beamergotobutton{Back}}}
% \begin{itemize}
%   \item Product: SAV 2.5K (second most popular product)
%   \item Impute prices:
%   \item \begin{enumerate}
%     \item RateWatch data is missing some weeks (around 75\% of observations have missing weeks)
%     $$r_{jwt} = \text{ avg non-missing }_{jt} + \text{ fed rate$_w$ } + \text{ national avg$_{jw}$,} $$
%     \item From a network of banks (rate setter) using RW.
%     \item Closest branch to the market centroid. 
%   \end{enumerate}
%   \item Drop bank-markets observations with less than 1\% rates.
%   \item Drop bank-markets observations with a single branch.
%   \item Local banks are defined as banks that get more than 90\% deposits from a single market. 
% \end{itemize}
% \end{frame}


% \begin{frame}[label=backg_col_brack_appendix]{Appendix\hyperlink{backg}{\beamergotobutton{Back}}}

% \end{frame}

\end{document}